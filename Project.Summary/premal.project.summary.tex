\documentclass[11pt,fleqn]{article}

\usepackage[letterpaper,left=1in,right=1in,top=1in,bottom=1in]{geometry}
\usepackage[singlespacing]{setspace}

%for NIH 
%\usepackage{helvet}
%\renewcommand{\familydefault}{\sfdefault} 

\usepackage[labelfont={footnotesize,bf},font={footnotesize}]{caption}
\usepackage{calc}
\usepackage{ifthen}
\usepackage{xspace}
\usepackage{graphicx}
\usepackage{color}
\usepackage{picinpar}
\usepackage{wrapfig} %alternative is floatflt -- Kopka and Daly 4th ed pg 179, floatfig is obsolete
%\usepackage{boxedminipage}
\usepackage{subfig}

\usepackage{amsfonts}
\usepackage{amsmath}
\usepackage{amssymb}
\usepackage{url}
\usepackage{xspace}
\usepackage{ifthen}
\usepackage{paralist} % allows for compact lists
\usepackage{afterpage} %for controlling float placement
\usepackage{placeins} %\FloatBarrier command 
\usepackage{float}  % for making custom floats and redefining float styles
\usepackage{multicol}
\usepackage{rotating}
\usepackage{sidecap} %SCfigure environment which puts captions on the side of a figure
\usepackage{fltpage} %FPfigure command ---put figure on following page had to install local
\usepackage{soul} % for underlining with \ul command which works much better than \underline
% See p 88 in Mittlebachand Goossen 2nd ed
\usepackage{verbatim}
%\usepackage{cite} %Enhanced references by number
\usepackage[numbers,sort&compress]{natbib}

\pagestyle{empty}

%
%From NSF's Grant Proposal Guidelines
%Use one of the following typefaces identified below:
%• Arial10, Courier New, or Palatino Linotype at a font size of 10 points or larger
%• Times New Roman at a font size of 11 points or larger
%• Computer Modern family of fonts at a font size of 11 points or larger
%A font size of less than 10 points may be used for mathematical formulas or equations, figure, table or diagram
%captions and when using a Symbol font to insert Greek letters or special characters. PIs are cautioned,
%however, that the text must still be readable;
%b. No more than 6 lines of text within a vertical space of 1 inch; and
%c. Margins, in all directions, must be at least an inch.
%These requirements apply to all uploaded sections of a proposal, including supplementary documentation.
%3. Page Formatting2
%Since many reviewers will be reviewing proposals electronically, proposers are strongly encouraged to use only
%a standard, single-column format for the text. Avoid using a two-column format since it can cause difficulties
%when reviewing the document electronically.
%While line spacing (single-spaced, double-spaced, etc.) is at the discretion of the proposer, established page
%limits must be followed. (Individual program solicitations, however, may eliminate this proposer option by
%requiring other type size, margin or line spacing requirements.)

\usepackage{soul} % for underlining with \ul command which works much better than \underline
\ifthenelse{1=1}{\singlespacing}{\doublespacing}

\begin{document}
\section*{Project Summary}
The number of  sequenced genomes currently stands at over 2000 and will continue to grow exponentially for the foreseeable future.
Each of these genomes is thought to contain a large amount of important information.
Extracting and interpreting this information is a major challenge in biology.
Our research will directly address this challenge using a novel, model based approach. 
Our research will allow biologists to extract the following information from an organism's genome: \ul{(a) codon specific missense and nonsense error rates and translational efficiencies, (b) a quantitative measure of the contribution these forces play in shaping the evolution of codon usage bias, and (c) reliable estimates of gene specific protein production rates.} % based solely on an organism's  genome.

%CUB is commonly observed across a wide range of organisms.
While numerous heuristic methods for measuring codon usage bias (CUB) exist, their biological interpretation is limited.
Our alternative approach to understanding CUB is based on mechanistic models of known biological processes. % (a model based approach, for brevity).
The model based approach we propose is \emph{highly innovative} and draws on the integration of  mathematical models from other disciplines such as molecular biology, population genetics, and Bayesian statistics.
%Model based approaches have the distinct advantages of greater statistical power and more biologically meaningful results.
Heretofore, widespread use of such models has been limited by the fact that they are more difficult to develop and implement.
However, with modern computer resources and statistical techniques, such approaches are now feasible.
Our previous success with these approaches and advances presented here clearly demonstrate the feasibility of our proposed research.


\subsection*{Intellectual Merits}
The contribution of our work to advancing our knowledge and understanding of biology spans many different fields.
For example, our current knowledge of error rates during protein translation is based on limited experimental data.% from molecular biologists.
%However, since these errors can act as a selective force on the evolution of CUB, the CUB observed across a genome should contain information on the rate and importance of these errors.
Our research will allow these researchers to access the addition information about these error rates held within an organism's genome.
A more precise understanding of the differences between missense and nonsense error rates will resolve a decades old uncertainty about their importance in the evolution of CUB.
Additionally, the vast majority of sequenced genomes are for non-model organisms whose biological roles are at best poorly understood.
Predicting gene expression levels is an important first step in our ability to draw inferences about an organism's ecological role using genomic data.
The ability of our approach to more precisely predict gene expression makes it of great utility to ecologists and systems biologists.
In conclusion, by dramatically increasing the amount and nature of information we can infer from a genome, our research will have a transformative effect on the fields of evolutionary, molecular, and systems biology.

%Identifying the set of genes an organism expresses and their average expression levels are important first steps in our ability to draw inferences about the ecological role such organisms play in their natural environment.

\subsection*{Broader Impacts}
The ability to infer gene expression from genomic data, will have numerous applications across a wide range of fields, including health care.
Through active mentoring by the PI and Co-PI for all participants, this project will provide postdoctoral, undergraduate and graduate students with hands on research experience.
Students will present results at local, regional, and national meetings.
Advanced students will be formally and informally involved in undergraduate teaching. 
Researchers will participate in various events such as scientific meetings, teacher collaboration programs, and summer REU program held at the National Institute for Mathematical and Biological Synthesis.
The PI will incorporate findings from this work into his undergraduate courses and other academic and public events.


\end{document}
  



\pagebreak

\section*{Project Summary}

\paragraph*{Research Description:}

The number of  sequenced genomes currently stands at over 2000 and will continue to grow exponentially for the foreseeable future.
Each of these genomes is thought to contain a large amount of important biological information.


\textbf{Our goal is to access this on fundamental biological process of protein translation using a combination of mechanistic models, population genetics, using Bayesian Inference.}
The outcome of our work will include (a) a quantitative understanding of the selective roles missense errors, nonsense errors, and translational efficiency play in the evolution of codon usage bias (CUB) and (b) reliable, genome wide predictions of gene specific protein production rates on an organism's the genomic sequence.
This information will be useful information about protein translation process, biology of organism,a nd resolve long standing debate about the relative importance these selective forces.
By dramatically increasing the amount and nature of information we can infer from a genome, our research will have a transformative effect on the fields of evolutionary, molecular, and systems biology.



The result of our work will outcome of our research will of our research is to use this information to quantify the understand the fundamental biological processes driving the evolution of


Our goal is to use this information to inform us about fundamental biological process and how the shape the evolution of the pattern of codon usage across a genome.
% genome structure the evolutionary forces driving them.
%that are thought to drive the evolution of codon bias.


Indeed, a number of different methods have been developed to link a gene's pattern of codon usage to its expression level.
However, as with most studies in bioinformatics, the basis of these methods are descriptive and associative offering no mechanism to explain the relationship.
Instead, we propose to base the analysis of DNA sequence data on mechanistic models of known biological processes (a model based approach, for brevity).
Because they require the integration of mathematical models from a wide range of disciplines such as molecular biology, population genetics, and statistics, model based approaches are more challenging than heuristic approaches.
Nevertheless, with modern computer resources and techniques, model based approaches are now feasible and represent an important example of how evolutionary models can be used to interpret the DNA sequence data.





The central goal of this project is to develop a model based approach which can be used to reliably predict the protein production rate for each gene within a genome using only DNA sequence data.
%We will reach this goal by taking a multi-scale and synthetic approach in which we
To reach the goal we draw on our current understanding of protein translation, gene fixation, and Monte-Carlo methods.
This project will be an extension of a newly developed Stochastic Evolutionary Model of Protein Production Rates (SEMPPR).
SEMPPR mechanistically links the pattern of synonymous codon usage within the coding sequence of a gene to that gene's protein production rate.
The model works by linking genotype to phenotype and phenotype to fitness and then uses a simple model of allele fixation to link fitness to fixation probability.
Simply put, SEMPPR's interpretation of sequence data rests on the general principle that genes which are expressed at a higher rate should show greater adaptation in reducing their production cost than genes expressed at a lower rate.
SEMPPR's current implementation requires a large number of species specific parameters, parameters which are only available for a handful of model organisms.
However, information on these parameters are reflected in the way synonymous codon usage is patterned across the genome.
This project will develop the means of extracting and applying this information. 
By extending SEMPPR, it will be possible to make genome wide, quantitative inferences about gene expression levels for the growing number of sequenced genomes.

\paragraph*{Intellectual Merits:}
The vast majority of these genomes are for non-model organisms whose biology are at best poorly understood.
Identifying the set of genes an organism expresses and their average expression levels are important first steps in our ability to draw inferences about the ecological role such organisms play in their natural environment.

Informing us about fundamental process of protein translation. %synthesis.


\subsection*{Broader Impacts}
A large proportion of sequenced genomes are human pathogens.
Our research will provide substantial information on the gene expression of these organisms and provide important clues to their ecology.
As a result this work has broad and important implications to society through its impact on disease ecology and epidemiology.
In addition, because SEMPPR works by calculating the selective pressure on synonymous substitutions, this work has broad implications for phylogenetic models which often assume such substitutions are neutral.
This project will also provide education for post-doctoral and graduate students in mathematical modeling, population genetics, and sequence analysis, with emphasis on their integration via modern statistical methods.
%Biology majors generally have a very limited appreciation of the central role mathematical models play in modern biology.
We propose to provide an opportunities for undergraduate students to learn about research in the seemingly disparate fields of mathematics and biology.
These educational goals will be complemented by activities at the National Institute for Mathematical and Biological Synthesis which was recently established at our home institution.

\end{document}

In addition, our model  nsequencGiven the explosiGiven that our underlying models are quite general and that many important human pathogens are now being sequenced, we expect our research to ultimately provide substantial information on the ecology and life-cycles of these important but poorly understood organisms.
Once the our research goal is achieved, we plan to apply our model to the set of available sequenced human pathogen and make these results publicly available. 
As a result, this research has broad and important \emph{benefits to society} through the fields of  medicine and epidemiology.


In addition to teaching within his Division and Department, the PI is also actively involved in the training of graduate students in the University's interdisciplinary graduate program on Genome Science and Technology.
Students in this program come from a diverse range of backgrounds and include many underrepresented groups, but generally lack a strong understanding of evolutionary processes and how they can be employed in data analysis.
The models underlying this work have been presented to these students as a clear and concrete example by which concepts in population genetics can be used as the starting point for interpreting bioinformatic datasets.
 
Finally, the students and PI will also engage in outreach through my lab's continued involvement in our local Darwin Day celebrations.
Our work provides a clear example of how evolutionary principles are an essential component of modern genome scale data analysis.
  
 \end{document}

The project will also provide a cohesive, ``hands on'' educational experience for graduate students in my laboratory who are being trained in interface of these areas as well as an introduction into computer based research for undergraduates majorng in Biology.



Gene expression, in the form of protein production rates, and adaptation are formally linked by applying Bayesian inference and mathematical models of protein translation and allele substitution to observed sequence data.


 can be used to determine reflect in the patternsThe ability to parameterize the model is based on the idea that 


the intragenic
We will reach this important goal through the application and testing of previously developed mathematical and computation techniques within a Bayesian statistical paradigm.
The \emph{expected outcome} of this work is the ability to make quantitative and testable predictions about the protein production rates for the expanding set of genomes, including many NIH model organisms such as \emph{Saccharomyces cerevisiae} and \emph{E.~coli}.


 
We believe that such questions are Reaching this 
Although generally more difficult to derive and implement, model based approaches have a number of distinct advantages over more heuristic approaches (see Box 1).
Towards this end, our group has recently developed a \emph{model based approach} for interpreting CUB through the use of a Stochastic Evolutionary Model of Protein Production Rates (SEMPPR) \citep{Gilchrist07}.
In this approach, CUB is linked directly to the micro-evolutionary forces of genetic drift, mutation and natural selection driving its evolution \citep{Bulmer91}.

  driven by computhe driving force behind most work While many algorithms and indices exist for analyzing sequence data, most approaches are computationally, rather than biologically, driven
Consequently, their results are generally difficult to interpret in a biologically meaningful manner.

While 
 extracting  in our ability to make  interpretable information from genome sequence data is one of the greatest challenges in the field of bioinformatics.
Developing reliable and quantitative tools for inferring gene expression levels from sequence data is one important example of this challenge.

Despite many years of research, understanding the roles selection, mutation, and genetic drift play in shaping an organism's phenotype remains a central goal in the well established field of evolutionary biology. 
Both theoretical and empirical evolutionary biologists believe that DNA sequence data contains fundamental information on the relative importance of these forces.
While progress has been made in specific systems, the major obstacle to extracting this information is the lack of a general quantitative framework which links genotype to phenotype and phenotype to fitness.
In parallel, extracting biologically important information from sequence data is a central goal in the emerging field of bioinformatics.
While many algorithms and indices exist for analyzing sequence data, most approaches are computationally, rather than biologically, driven
Consequently, their results are generally difficult to interpret in a biologically meaningful manner.
%Beyond the difficulties in interpretating these results, the lack of biologically oriented models means that the type of information encoded within sequence data and how it is best extracted remains unknown.
 As a result, despite massive data collection efforts, both evolutionary biologists and bioinformaticians find themselves struggling to address questions central to their respective fields.
The general solution to this problem is develop bioinformatics techniques which are based on the integration of mechanistic models, statistical inference, and  population genetics.

% LocalWords:  nsequencGiven explosiGiven bioinformatic majorng patternsThe
% LocalWords:  computhe bioinformaticians
