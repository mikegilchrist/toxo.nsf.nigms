\documentclass[11pt,fleqn,letterpaper]{article}


\usepackage[letterpaper,left=1in,right=1in,top=1in,bottom=1in]{geometry}
\usepackage[singlespacing]{setspace}

%for NIH 
%\usepackage{helvet}
%\renewcommand{\familydefault}{\sfdefault} 

\usepackage[labelfont={footnotesize,bf},font={footnotesize}]{caption}
\usepackage{calc}
\usepackage{ifthen}
\usepackage{xspace}
\usepackage{graphicx}
\usepackage{color}
\usepackage{picinpar}
\usepackage{wrapfig} %alternative is floatflt -- Kopka and Daly 4th ed pg 179, floatfig is obsolete
%\usepackage{boxedminipage}
\usepackage{subfig}

\usepackage{amsfonts}
\usepackage{amsmath}
\usepackage{amssymb}
\usepackage{url}
\usepackage{xspace}
\usepackage{ifthen}
\usepackage{paralist} % allows for compact lists
\usepackage{afterpage} %for controlling float placement
\usepackage{placeins} %\FloatBarrier command 
\usepackage{float}  % for making custom floats and redefining float styles
\usepackage{multicol}
\usepackage{rotating}
\usepackage{sidecap} %SCfigure environment which puts captions on the side of a figure
\usepackage{fltpage} %FPfigure command ---put figure on following page had to install local
\usepackage{soul} % for underlining with \ul command which works much better than \underline
% See p 88 in Mittlebachand Goossen 2nd ed
\usepackage{verbatim}
%\usepackage{cite} %Enhanced references by number
\usepackage[numbers,sort&compress]{natbib}

\pagestyle{empty}

%
%From NSF's Grant Proposal Guidelines
%Use one of the following typefaces identified below:
%• Arial10, Courier New, or Palatino Linotype at a font size of 10 points or larger
%• Times New Roman at a font size of 11 points or larger
%• Computer Modern family of fonts at a font size of 11 points or larger
%A font size of less than 10 points may be used for mathematical formulas or equations, figure, table or diagram
%captions and when using a Symbol font to insert Greek letters or special characters. PIs are cautioned,
%however, that the text must still be readable;
%b. No more than 6 lines of text within a vertical space of 1 inch; and
%c. Margins, in all directions, must be at least an inch.
%These requirements apply to all uploaded sections of a proposal, including supplementary documentation.
%3. Page Formatting2
%Since many reviewers will be reviewing proposals electronically, proposers are strongly encouraged to use only
%a standard, single-column format for the text. Avoid using a two-column format since it can cause difficulties
%when reviewing the document electronically.
%While line spacing (single-spaced, double-spaced, etc.) is at the discretion of the proposer, established page
%limits must be followed. (Individual program solicitations, however, may eliminate this proposer option by
%requiring other type size, margin or line spacing requirements.)


\usepackage{tabularx} % provides nice text wrapping in tables.
\ifthenelse{1=1}{\singlespacing}{\doublespacing}

\graphicspath{{./Figures/}}
\DeclareGraphicsExtensions{.pdf,.png}

\renewcommand\thesection {\Alph{section}}
\renewcommand\refname{Literature Cited}



\setlength{\mathindent}{0in}

%\setcounter{page}{1} 

%%local commands
\newcommand{\toxo}{\emph{T.~gondii}\xspace}
\newcommand{\toxog}{\emph{Toxoplasma gondii}\xspace}
\newcommand{\invitro}{\emph{in vivo}\xspace}
\newcommand{\invivo}{\emph{in vitro}\xspace}
\newcommand{\setL}{\ensuremath{\mathbf{\Lambda}}\xspace}
\newcommand{\Lik}{\ensuremath{\text{Lik}}\xspace}
\newcommand{\LLik}{\ensuremath{\text{LLik}}\xspace}
\newcommand{\IFG}{IF$\gamma$\xspace}
\newcommand{\smini}{\ensuremath{s_{i,\min}}\xspace}
\newcommand{\smint}{\ensuremath{s_{T,\min}}\xspace}
\newcommand{\sminb}{\ensuremath{s_{B,\min}}\xspace}
\newcommand{\smaxi}{\ensuremath{{s_{i,\max}}}\xspace}
\newcommand{\smaxt}{\ensuremath{{s_{T,\max}}}\xspace}
\newcommand{\smaxb}{\ensuremath{{s_{B,\max}}}\xspace}


%%sidebox numbers
\newcommand{\testBox}{0\xspace}


%%define function for putting comments in the margin
\setlength{\marginparwidth}{0.75in}
\newcommand\myMarginPar[1]{\marginpar{\begin{spacing}{0.7}\singlespacing \tiny #1 \end{spacing}}}
%I don't think the being{spacing} is having any effect
\ifthenelse{1=0}{
\newcommand{\printdraft}{true}
}{
\newcommand{\printdraft}{false}

}

\setlength{\mathindent}{0.125in}

\setlength{\intextsep}{3 pt} %originally 3pt was set to 0 , sets space surrounding floats inserted using wrapfig


%%define function for making boxes of text and math
\newcounter{sideboxctr}
\renewcommand{\thesideboxctr}{\arabic{sideboxctr}}

\ifthenelse{1=1}{

  % to get box automatically put around text  
  \newsavebox{\ftextbox} 
  \newlength{\ftextwidth}

  \newenvironment{sidebox}[3][c]% %tried adding wrapfigure commands but that screwed things up
  % first two arguments are after wrapfigure
  % \sidebox[pos]{width}{title}.  
  % Would like to figure out how to use \label command. Think I need to create a custom float but then it wouldn't work with wrapfigure
  { %opening definition
    \stepcounter{sideboxctr} %to make material boxed automatically check out Mittlebach & Goosens 2nd ed pg 870
    \setlength{\ftextwidth}{#2-4\fboxsep-4\fboxrule}%define variable value based on desired length
    \begin{lrbox}{\ftextbox}%
      \hspace*{\fboxsep}\hspace*{\fboxrule}%      \begin{center}%
      \begin{minipage}[#1]{\ftextwidth}
        \setlength{\parindent}{5mm}
        \vspace*{4pt}%
        \small%
        \begin{center}
          \textbf{Box \thesideboxctr: #3}\\
          \vspace*{-2\lineskip} %was -4\lineskip
          \rule[\lineskip]{0.66\linewidth}{0.4pt}%\rule[lift]{width}{height}
        \end{center}
          \vspace*{-10\lineskip} %was -12\lineskip
          \noindent 
      }
      { %closing definition
      \end{minipage}
      % \end{center}
      %\vspace*{4pt}
    \end{lrbox}\fbox{\usebox{\ftextbox}}
  }
}
{

   %define sidebox float -- would allow use of \label 
   % should automatically define a new wrapsidebox environment but doesn't seem to work
   % doesn't work with wrapfigure since it is not a ``figure'' float 
   % \floatstyle{boxed}
   \newfloat{sidebox}{tbhp}{lob} %{name}{placement}{aux file to write to}
   \floatname{sidebox}{Box}
 }

\floatstyle{plain}
\newfloat{mytextfloat}{tbh!}{lob}%[subsection] %{name}{placement}{aux file to write to}[within structure]
\floatname{mytextfloat}{}



%%Redefine paragraph
%%taken from /usr/share/texmf-texlive/tex/latex/base/article.cls
%\ifthenelse{1=1}{
  \makeatletter
  \renewcommand{\section}{\@startsection {section}{1}{\z@}%
    {-3ex \@plus -0.55ex \@minus -.14ex}%
    {1ex \@plus.8ex}%
    {\normalfont\Large\bfseries}}
  \renewcommand{\subsection}{\@startsection{subsection}{2}{\z@}%
    {-3ex\@plus -0.4ex \@minus -.1ex}%
    {0.4ex \@plus .071ex}%
    {\normalfont\large\bfseries}}
  \renewcommand{\paragraph}{\@startsection{paragraph}{4}{\z@}% name, level, indent
    {-0.3\baselineskip \@plus0.25\baselineskip \@minus.1\baselineskip}% beforeskip
    {-1em}%afterskip
    {\normalfont\normalsize\bfseries}}
  \makeatother 
%}{}



\begin{document}

\section*{Results from Prior NSF Support}




\section{Objectives}

Our Specific Aims are to 
\begin{enumerate} 
\item Develop and parameterize models of \toxo behavior for epithelial, muscle, neuron, and [immune response cell] \invitro
\item Combine cell population model into a cell community model to describe organ level processes and (b) link gut, muscle, brain, and vascular organs together.
Use to guide and fit to \invivo data.
\item Between Host Model
\end{enumerate} 

\textbf{A direct outcome of our research will be the ability to \ldots}



\section{Background}




\section{Preliminary Results} 

\section{Research Design \& Methods} \label{sec:research.design}
In order to develop an cohesive and biologically based model of \toxo infection at the within and between host scales, we combine and nest different models together.
%The resulting framework describes the dynamics of an infection at the intracellular, tissue, organ, host, and epidemilogical scales.
This separation of modeling scales allows us to both develop and parameterize a model that takes into account intracellular, tissue, organ, host, and epidemilogical scale processes.

For example, in Specific Aim 1 we begin by constructing a simple tissue level model.
The tissue level model uses ordinary and partial differential equations (ODE and PDE, respectively) to model the dynamics of the parasite, uninfected and infected host cells of a single type (e.g.~epilthelial or muscle cells), and an indicator of the immune response state of the local tissue such as \IFG level (Figure \ref{figTissueLevel}).
As part of Specific Aim 2, our tissue level model will be expanded to include multiple cell types and, thus, model the dynamics at the level of an `organ' such as the gut, skeletal muscle, or brain.
In addition, these multiple organ level models will, in turn, be coupled together to model the dynamics of the infection at the individual host scale to varying degrees of sophistication (Figure \ref{figCombinedLevels}).
Finally, for Specific Aim 3 \ldots


  
\subsection*{Specific Aim 1: Development and Parameterization of Tissue Model }
\paragraph*{Tissue Model Formulation:} %%%PUT MODEL IN BOX?
In order to describe the dynamics of an infection at the level of a single tissue (i.e.~where there is only one host cell type), 
%we use ODEs to model the dynamics of the uninfected host cells, free parasites, and host immune response level and PDEs to describe the populations of host cells infected by tachyzoite or bradyzoites.
we begin by assuming that in the absence of the infection, the density of uninfected cells $X_0$ is determined by a density dependent production rate $r(X_0)$ and background mortality $d_X$ and defining the time scale as $a$ where, in general $a_0$ is time at which an infected cell first arrived in the focal tissue.
(See Table \ref{tableTissueDefs} for definitions of all symbols used in the tissue level model.)
We classify infected host cells based on whether they contain \toxo in the tachyzoite or bradyzoite state which we indicate using the subscript $i \in \{T, B\}$, respectively,
Thus, we let $X_T$ and $X_B$ represent the densities of host cells infected by a tachyzoite containing PV or a bradyzoite containing cyst.
We structure the infected cells $X_T$ and $X_B$ by both the age of infection $a$ and size of the parasite population within the cell $s$ and use PDEs to describe their behavior.
Because the overall sizes of a tachyzoite and bradyzoites are similar, we can estimate $s$ empirically for any given cell based on the the volume of its PV or cyst.
We assume that $s$ grows according to the growth functions $g_T$ and $g_B$ which can be affected by both $s$ and other factors, such as the immune response state of the tissue $Z$ (see below).
The densities of infected cells $X_T$ and $X_B$ are reduced by the background cell removal rate $d_X$ as well as PV and cyst bursting rates $b_T$ and $b_B$, respectively.
Cell bursting results in the release of free parasites into the tissue.
We assume that $b_T$ and $b_B$ can vary between cell types, parasite strain, and $Z$.
For example, $b_T$ may increase rapidly with $s$, thus effectively limiting the maximum size of a PV, while $b_B$ may stay relatively constant over a large range of $s$ values.
The densities of infected cells are also altered by stage conversion of tachyzoites into bradyzoites within the cell.
We model this as the conversion of $X_T(a,s)$ into $X_B(a,s)$ which occurs at rate $c_{TB}$.
As with bursting, we let $c_{TB}$ vary depending on the cell type, parasite strain, $s$, and $Z$.
For example, the conversion rate from $X_T$ to $X_B$ in epithelial tissues is negligible while the conversion rate in muscle tissues is substantial.
Further, in general we expect $c_{TB}$ to increase with the size $s$ of the PV.
Based on experimental data [CITATION], we assume that immune response clearance of infected cells is negligible.


Experimental data also indicates that bradyzoites rapidly convert to tachyzoites right after cell bursting [CITATION].
As a result, we assume that bradyzoites released when $X_B$ cells burst are immediately converted into to free tachyzoites.
This means that all newly infected cells are initially in the $X_T$ class and that any $X_B$ cells are the result of stage conversion.
Uninfected cells $X_0$ become infected when a free parasite $P$ encounters, enters, and successfully establishes itself within the host cell.
Because the immune response state of the tissue $Z$ can alter the state of the host cell, interfering with the parasite's ability to establishment an infection, we define $h$ as the probability a newly ingressed parasite is able successfully infect a host cell and assume it is a function of $Z$.
Free parasites are also removed from the tissue at a background mortality rate $d_P$ and, unlike infected cells, through immune response mediated mortality $d_{PZ}$.

Regarding the immune response $Z$, modeling the immune response of a host to a pathogen could be an entire research program in and of itself and is outside the scope of this proposal.
Thus, for our work we treat the immune response dynamics as a forcing function $f(a)$ whose stereotypical behavior is determined by the host genotype, parasite strain, and tissue type and location.
Thus the dynamics of $Z(a)$ are largely independent from the dynamics of the other, non-immune response model components. 
[NEED ADDITIONAL DISCUSSION OF HOW WE WILL RELATE DIFFERENT COMPONENTS OF THE IMMUNE RESPONSE TO EACH OTHER]
Information on immune response dynamics already exist at various scales, including the entire host as well as specific tissues and organs  such as muscle [CITATIONS], blood [CITATION], and brain [CITATIONS]
In our \invitro experiments, only a single aspect of the immune response will be mainipulated at a time.
As we combine models in later formulations, we will include multiple components.

More formally, our tissue level model can be written as,
\begin{align*}
\frac{d X_0}{dt} &= \lambda\left(X_0\right) - \left(k_{PX} P + d_X\right) X_0\\
\frac{d P}{dt} &= \sum_{i \in \{T,B\}} \int_1^\smaxi s \, b_i(s, Z) X_i\left(a, s\right) ds - \left(k_{PX} X_0 + d_P + d_{PZ}(Z)\right) P \label{eqParsiteODE}\\
\frac{\partial X_T}{\partial a} &+ \frac{\partial }{\partial s} \left( g_T(s) X_B(a,s)\right) = - \left(d_X - c_{TB}(s, Z)\right) X_T(a,s) \label{eqTachyPDE}\\
\frac{\partial X_B}{\partial a} &+ \frac{\partial }{\partial s} \left( g_B(s) X_B(a,s)\right) = - \left(d_X + c_{TB}(s,Z)\right) X_T(a,s)\\
\frac{d Z}{dt} &=f(a)\\
\intertext{with a boundary condition for Equation (\ref{eqTachyPDE}) of}
h(Z) k_{PX} X_0(a) &= g\left(\smint\right) X_T\left(a, \smint\right).
\end{align*}
Where \smaxi ($i \in \{T,B\}$) represents the maximum PV or cyst size possible, a value indirectly determined by the growth functions $g_T$ and $g_B$ and the host cell mortality rate $d_X$.
The integration in Eq.~(\ref{eqParasiteODE}) quantifies the rate at which parasites are being released from infected cells $X_i, i \in \{B, T\}$ by weighting the bursting rate of infected cells by the size of the parasite population released $s$.
The first differentials in Eqs.~(\ref{eqTachyPDE})  and (\ref{eqTachyPDE}) describe the `aging' of an infected cell while the second differentials describes how changes in cell density and parasite population growth rates, $g_i$, affect the change in density of cells.


\begin{table}
\begin{tabularx}{\linewidth}{ r X } % X indicates column where text wraps around
Symbol & Definition\\ \hline
$a$ & Age of the infection time scale\\
$s$ & Size of parasite population within an infected cell.\\
$X_0(a)$ & Density of uninfected host cells at time  $a$\\
$P(a)$ & Density of free tachyzoites at time $a$\\
$X_T(a,s)$ & Density of host cells infected by tachyzoites population of size $s$ at time $a$. \\
$X_B(a,s)$ & Density of host cells infected by bradyzoites population of size $s$ at time $a$. \\
$Z(a)$ & Immune response level at time $a$\\
$b_T(s, Z)$& Bursting rate of tachyzoite containing cells $X_T$ as a function of $s$ and $Z$.\\
$b_B(s, Z)$& Bursting rate of bradyzoite containing cells $X_T$ as a function of $s$ and $Z$.\\
$k_{PX}$ & Rate at which free parasite $P$ encounter uninfected cells $X_0$.\\
$g_T, g_B$ & Intracellular population growth rate of tachyzoite and bradyzoites, respectively.\\
$h(Z)$ & Probability of PV formation by an invading tachyzoite as a function of $Z$.\\
$d_X$ \& $d_P$ & Background mortality rates of host cells $X$ and free tachyzoites $P$, respectively.\\
$d_{PZ}(Z)$ & Immune response mediated clearance of free tachyzoites.\\
$c_{TB}$ & Conversion rate of tachyzoite containing cells $X_T$ into bradyzoite containing cells $X_B$.
\end{tabularx}
\caption{Symbol definitions for tissue level model.
  Could reduce text size and/or column width and then wrap text around this to save space.
}
\end{table}

\paragraph*{Tissue Model Parameterization:}
Because of our  tissue model assumes there's only one cell type, we can use time series data collected from \invitro experiments using cell cultures to parameterize key aspects of it.
For example, by analyzing samples of from the cell culture, we will be able to estimate
\begin{enumerate}
\item Bursting rate $b_i, i \in \{T, B\}$
\item Conversion rate rate $c_{TB}$
\item PV formation probability $h(Z)$
\end{enumerate}

\subsection*{Specific Aim 2: Development of Organ and Whole Body Model}
We scale up our tissue level model by first expanding it into a organ scale model in which there are multiple host cell types.
Next, we link multiple organ level models together to create a simple whole body model of a \toxog infection.

\paragraph*{Organ Level Model Formulation:}
We can expand our tissue model into an organ level model by retaining all of the previous assumptions and increasing the number of host target cell types.
Each additional host target cell type introduces three additional equations, one for the uninfected cells and one for each category of infected cells, tachyzoite and bradyzoite.


\paragraph*{Whole Body Model Formulation}


\paragraph*{Organ \& Whole Body Model Parameterization}





%%%%%%% BEGIN BOX 3 %%%%%%%%%%%%%%%%%%%%
\begin{mytextfloat}[!b]
\vspace*{-2ex}
  \begin{sidebox}{\textwidth}{Calculating the Likelihood of a Parameter Set \setL}
    
   \vspace*{-0.3in}
    
    \begin{multicols}{2}
\noindent If we are interested in first estimating the set of parameters \setL used to calculate $\eta$ we can treat the protein production rate of a gene $\phi$ as a nuisance parameter and integrate over it.
Combining Eqn.~(\ref{eq:pr.eta.I} with the assumption that $\phi \sim \text{Exp}(\zeta)$, the probability of observing a sequence fixed in a population as a function of \setL is,
\vspace*{-0.5ex}
\begin{equation}\label{eq:pr.eta.II}
a + b = c
\end{equation}
Formally, Eqn.~(\ref{eq:pr.eta.II}) represents the likelihood of \setL given a codon sequence.
Thus, the total likelihood of \setL given the $n$ observed gene sequences within a genome is simply,
\begin{equation}
  \label{eq:totallik}
  \Lik(\setL|Z) = \prod_{i=1}^n P(Z| \setL).
\end{equation}
Whether likelihood function of Eqn.~(\ref{eq:totallik}) is analyzed directly or, as we propose, weighted by a prior distribution, it provides method for estimating \setL using genomic data alone.
These \setL values, in turn, can be used to predict $\phi$ (see Figure \ref{fig:mcmc}).
%\vspace*{1ex}
   \end{multicols}
   \vspace*{-1ex}
  \end{sidebox}
\end{mytextfloat}
%%%%%%% END BOX 3 %%%%%%%%%%%%%%%%%%%%


\subsection*{Specific Aim 2: Organ and Within-Host Modeling}

\begin{wrapfigure}{r}{0.5\textwidth}
\fbox{
%\includegraphics[width=0.92\linewidth]{Figures/phi-obs-vs-pred-mcmc}
}\caption{Illustration of SEMPPR's ability to accurately predict $\phi$ using parameters \setL generated from a simple MCMC algorithm and Eqn.~(\ref{eq:totallik}).}\label{fig:mcmc}
\end{wrapfigure}


\subsection*{Limitations to our Work}\label{sec:limitations}


\subsection*{Timeline of Proposed Research}

\subsection*{Relationship of Proposed Work to Long-Term Goals}



\section{Broader Impacts}\label{sec:broader.impacts} 

\pagebreak

\setcounter{page}{1} 
%\bibliographystyle{acm}
%\bibliographystyle{plain}
\bibliographystyle{/home/mikeg/BiBTeX/am.nat}
\bibliography{./bibliography}

\end{document}

