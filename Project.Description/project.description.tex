\documentclass[11pt,fleqn,letterpaper]{article}


\usepackage[letterpaper,left=1in,right=1in,top=1in,bottom=1in]{geometry}
\usepackage[singlespacing]{setspace}

%for NIH 
%\usepackage{helvet}
%\renewcommand{\familydefault}{\sfdefault} 

\usepackage[labelfont={footnotesize,bf},font={footnotesize}]{caption}
\usepackage{calc}
\usepackage{ifthen}
\usepackage{xspace}
\usepackage{graphicx}
\usepackage{color}
\usepackage{picinpar}
\usepackage{wrapfig} %alternative is floatflt -- Kopka and Daly 4th ed pg 179, floatfig is obsolete
%\usepackage{boxedminipage}
\usepackage{subfig}

\usepackage{amsfonts}
\usepackage{amsmath}
\usepackage{amssymb}
\usepackage{url}
\usepackage{xspace}
\usepackage{ifthen}
\usepackage{paralist} % allows for compact lists
\usepackage{afterpage} %for controlling float placement
\usepackage{placeins} %\FloatBarrier command 
\usepackage{float}  % for making custom floats and redefining float styles
\usepackage{multicol}
\usepackage{rotating}
\usepackage{sidecap} %SCfigure environment which puts captions on the side of a figure
\usepackage{fltpage} %FPfigure command ---put figure on following page had to install local
\usepackage{soul} % for underlining with \ul command which works much better than \underline
% See p 88 in Mittlebachand Goossen 2nd ed
\usepackage{verbatim}
%\usepackage{cite} %Enhanced references by number
\usepackage[numbers,sort&compress]{natbib}

\pagestyle{empty}

%
%From NSF's Grant Proposal Guidelines
%Use one of the following typefaces identified below:
%• Arial10, Courier New, or Palatino Linotype at a font size of 10 points or larger
%• Times New Roman at a font size of 11 points or larger
%• Computer Modern family of fonts at a font size of 11 points or larger
%A font size of less than 10 points may be used for mathematical formulas or equations, figure, table or diagram
%captions and when using a Symbol font to insert Greek letters or special characters. PIs are cautioned,
%however, that the text must still be readable;
%b. No more than 6 lines of text within a vertical space of 1 inch; and
%c. Margins, in all directions, must be at least an inch.
%These requirements apply to all uploaded sections of a proposal, including supplementary documentation.
%3. Page Formatting2
%Since many reviewers will be reviewing proposals electronically, proposers are strongly encouraged to use only
%a standard, single-column format for the text. Avoid using a two-column format since it can cause difficulties
%when reviewing the document electronically.
%While line spacing (single-spaced, double-spaced, etc.) is at the discretion of the proposer, established page
%limits must be followed. (Individual program solicitations, however, may eliminate this proposer option by
%requiring other type size, margin or line spacing requirements.)


\ifthenelse{1=1}{\singlespacing}{\doublespacing}

\graphicspath{{./Figures/}}
\DeclareGraphicsExtensions{.pdf,.png}

\renewcommand\thesection {\Alph{section}}
\renewcommand\refname{Literature Cited}



\setlength{\mathindent}{0in}

%\setcounter{page}{1} 

%%local commands
\newcommand{\toxo}{\emph{T.~gondii}\xspace}
\newcommand{\toxog}{\emph{Toxoplasma gondii}\xspace}
\newcommand{\invitro}{\emph{in vivo}\xspace}
\newcommand{\invivo}{\emph{in vitro}\xspace}


%%sidebox numbers
\newcommand{\testBox}{0\xspace}


%%define function for putting comments in the margin
\setlength{\marginparwidth}{0.75in}
\newcommand\myMarginPar[1]{\marginpar{\begin{spacing}{0.7}\singlespacing \tiny #1 \end{spacing}}}
%I don't think the being{spacing} is having any effect
\ifthenelse{1=0}{
\newcommand{\printdraft}{true}
}{
\newcommand{\printdraft}{false}

}

\setlength{\mathindent}{0.125in}

\setlength{\intextsep}{3 pt} %originally 3pt was set to 0 , sets space surrounding floats inserted using wrapfig


%%define function for making boxes of text and math
\newcounter{sideboxctr}
\renewcommand{\thesideboxctr}{\arabic{sideboxctr}}

\ifthenelse{1=1}{

  % to get box automatically put around text  
  \newsavebox{\ftextbox} 
  \newlength{\ftextwidth}

  \newenvironment{sidebox}[3][c]% %tried adding wrapfigure commands but that screwed things up
  % first two arguments are after wrapfigure
  % \sidebox[pos]{width}{title}.  
  % Would like to figure out how to use \label command. Think I need to create a custom float but then it wouldn't work with wrapfigure
  { %opening definition
    \stepcounter{sideboxctr} %to make material boxed automatically check out Mittlebach & Goosens 2nd ed pg 870
    \setlength{\ftextwidth}{#2-4\fboxsep-4\fboxrule}%define variable value based on desired length
    \begin{lrbox}{\ftextbox}%
      \hspace*{\fboxsep}\hspace*{\fboxrule}%      \begin{center}%
      \begin{minipage}[#1]{\ftextwidth}
        \setlength{\parindent}{5mm}
        \vspace*{4pt}%
        \small%
        \begin{center}
          \textbf{Box \thesideboxctr: #3}\\
          \vspace*{-2\lineskip} %was -4\lineskip
          \rule[\lineskip]{0.66\linewidth}{0.4pt}%\rule[lift]{width}{height}
        \end{center}
          \vspace*{-10\lineskip} %was -12\lineskip
          \noindent 
      }
      { %closing definition
      \end{minipage}
      % \end{center}
      %\vspace*{4pt}
    \end{lrbox}\fbox{\usebox{\ftextbox}}
  }
}
{

   %define sidebox float -- would allow use of \label 
   % should automatically define a new wrapsidebox environment but doesn't seem to work
   % doesn't work with wrapfigure since it is not a ``figure'' float 
   % \floatstyle{boxed}
   \newfloat{sidebox}{tbhp}{lob} %{name}{placement}{aux file to write to}
   \floatname{sidebox}{Box}
 }

\floatstyle{plain}
\newfloat{mytextfloat}{tbh!}{lob}%[subsection] %{name}{placement}{aux file to write to}[within structure]
\floatname{mytextfloat}{}



%%Redefine paragraph
%%taken from /usr/share/texmf-texlive/tex/latex/base/article.cls
%\ifthenelse{1=1}{
  \makeatletter
  \renewcommand{\section}{\@startsection {section}{1}{\z@}%
    {-3ex \@plus -0.55ex \@minus -.14ex}%
    {1ex \@plus.8ex}%
    {\normalfont\Large\bfseries}}
  \renewcommand{\subsection}{\@startsection{subsection}{2}{\z@}%
    {-3ex\@plus -0.4ex \@minus -.1ex}%
    {0.4ex \@plus .071ex}%
    {\normalfont\large\bfseries}}
  \renewcommand{\paragraph}{\@startsection{paragraph}{4}{\z@}% name, level, indent
    {-0.3\baselineskip \@plus0.25\baselineskip \@minus.1\baselineskip}% beforeskip
    {-1em}%afterskip
    {\normalfont\normalsize\bfseries}}
  \makeatother 
%}{
%}



\begin{document}

\section*{Results from Prior NSF Support}




\section{Objectives}

Our Specific Aims are to 
\begin{enumerate} %\begin{compactenum}
\item Develop and parameterize models of \toxo behavior for epithelial, muscle, neuron, and [immune response cell] \invitro
\item Combine cell population model into a cell community model to describe organ level processes and (b) link gut, muscle, brain, and vascular organs together.
Use to guide and fit to \invivo data.
\item Between Host Model
\end{enumerate} 
\textbf{A direct outcome of our research will be the ability to \ldots}



\section{Background}




\section{Preliminary Results} 

\section{Research Design \& Methods} \label{sec:research.design}
\subsection*{Specific Aim 1:}
Text



%%%%%%% BEGIN BOX 3 %%%%%%%%%%%%%%%%%%%%
\begin{mytextfloat}[!b]
\vspace*{-2ex}
  \begin{sidebox}{\textwidth}{Calculating the Likelihood of a Parameter Set \setL}
    
   \vspace*{-0.3in}
    
    \begin{multicols}{2}
\noindent If we are interested in first estimating the set of parameters \setL used to calculate $\eta$ we can treat the protein production rate of a gene $\phi$ as a nuisance parameter and integrate over it.
Combining Eqn.~(\ref{eq:pr.eta.I} with the assumption that $\phi \sim \text{Exp}(\zeta)$, the probability of observing a sequence fixed in a population as a function of \setL is,
\vspace*{-0.5ex}
\begin{equation}\label{eq:pr.eta.II}
  P(\cvec| \setL) = \frac{\zeta' \,e^{\beta\left(\zeta' + \beta\right)} \Gamma\left(\alpha+1, \beta\left(\zeta' + \eta(\cvec)\right)\right)}{|\setC| N_e q \beta^{-\alpha} \left(\zeta' + \eta(\cvec)\right)^{\alpha - 1} }
\end{equation}
 where $\zeta' = \zeta/(q N_e)$ and $\eta(\cvec)$, $\alpha$, and $\beta$ are all implicit functions of \setL.
Formally, Eqn.~(\ref{eq:pr.eta.II}) represents the likelihood of \setL given a codon sequence \cvec.
Thus, the total likelihood of \setL given the $n$ observed gene sequences within a genome \cobsvec is simply,
\begin{equation}
  \label{eq:totallik}
  \Lik(\setL|\cobsvec) = \prod_{i=1}^n P(\cobsnum{i}| \setL).
\end{equation}
Whether likelihood function of Eqn.~(\ref{eq:totallik}) is analyzed directly or, as we propose, weighted by a prior distribution, it provides method for estimating \setL using genomic data alone.
These \setL values, in turn, can be used to predict $\phi$ (see Figure \ref{fig:mcmc}).
%\vspace*{1ex}
   \end{multicols}
   \vspace*{-1ex}
  \end{sidebox}
\end{mytextfloat}
%%%%%%% END BOX 3 %%%%%%%%%%%%%%%%%%%%


\subsection*{Specific Aim 2: Organ and Within-Host Modeling}
\begin{wrapfigure}{r}{0.5\textwidth}
\fbox{
\includegraphics[width=0.92\linewidth]{Figures/phi-obs-vs-pred-mcmc}
}\caption{Illustration of SEMPPR's ability to accurately predict $\phi$ using parameters \setL generated from a simple MCMC algorithm and Eqn.~(\ref{eq:totallik}).}\label{fig:mcmc}
\end{wrapfigure}


\subsection*{Limitations to our Work}\label{sec:limitations}


\subsection*{Timeline of Proposed Research}

\subsection*{Relationship of Proposed Work to Long-Term Goals}



\section{Broader Impacts}\label{sec:broader.impacts} 

\pagebreak

\setcounter{page}{1} 
%\bibliographystyle{acm}
%\bibliographystyle{plain}
\bibliographystyle{/home/mikeg/BiBTeX/am.nat}
\bibliography{./bibliography}

\end{document}

